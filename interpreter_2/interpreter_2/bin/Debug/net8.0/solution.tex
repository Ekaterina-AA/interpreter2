\documentclass{article}
\usepackage[utf8]{inputenc}
\usepackage[russian]{babel}
\usepackage{amsmath}
\usepackage{amssymb}
\begin{document}
\title{Решение задач}
\maketitle
\section{Задача: Скалярное произведение}
\subsection{Исходные данные}
(1, 2, 3);(4, 5, 6)
\subsection{Решение}
\begin{align*}
\mathbf{a} \cdot \mathbf{b} &= 
1 \times 4 + \\ &+ 
2 \times 5 + \\ &+ 
3 \times 6 = \\
&= 4 + 10 + 18 \\
&= 32
\end{align*}
\section{Задача: Векторное произведение}
\subsection{Исходные данные}
(1, 0, 0);(0, 1, 0)
\subsection{Решение}
\begin{align*}
\mathbf{a} \times \mathbf{b} &= \begin{vmatrix} \mathbf{i} & \mathbf{j} & \mathbf{k} \\ 1 & 0 & 0 \\ 0 & 1 & 0 \end{vmatrix} \\
&= \mathbf{i}(0 \times 0 - 0 \times 1) \\
&- \mathbf{j}(1 \times 0 - 0 \times 0) \\
&+ \mathbf{k}(1 \times 1 - 0 \times 0) \\
&= (0, 0, 1)
\end{align*}
\section{Задача: Смешанное произведение}
\subsection{Исходные данные}
(1, 1, 1);(2, 2, 2);(3, 3, 3)
\subsection{Решение}
\begin{align*}
[\mathbf{a}, \mathbf{b}, \mathbf{c}] &= (\mathbf{a} \times \mathbf{b}) \cdot \mathbf{c} \\
&= \left(\begin{vmatrix} 1 & 1 \\ 2 & 2 \end{vmatrix}, -\begin{vmatrix} 1 & 1 \\ 2 & 2 \end{vmatrix}, \begin{vmatrix} 1 & 1 \\ 2 & 2 \end{vmatrix}\right) \cdot (3, 3, 3) \\
&= (0 \times 3) + (0 \times 3) + (0 \times 3) \\
&= 0
\end{align*}
\section{Задача: Арифметические операции над векторами}
\subsection{Исходные данные}
+:3:(1, 4, 1);(6, 2, 2);(5, 3, 8)
\subsection{Решение}
\begin{align*}
\mathbf{v}_1 + \mathbf{v}_2 &= (1, 4, 1) + (6, 2, 2) \\&= (7, 6, 3)
\end{align*}
\section{Задача: Модуль вектора}
\subsection{Исходные данные}
(1, 2, 3)
\subsection{Решение}
\begin{align*}
\|\mathbf{v}\| &= \sqrt{ 1^2 + 2^2 + 3^2 } \\
&= \sqrt{ 14 } \\
&= 3,7416573867739413
\end{align*}
\section{Задача: Процесс ортогонализации}
\subsection{Исходные данные}
3:(0, 1, 0);(2, 0, 2);(2, 3, 1)
\subsection{Решение}
\begin{align*}
\text{Процесс ортогонализации Грама-Шмидта:}\\
\mathbf{u}_1 &= (0, 1, 0) \\
\mathbf{u}_2 &= \mathbf{v}_2 - \frac{(\mathbf{v}_2 \cdot \mathbf{u}_1)}{(\mathbf{u}_1 \cdot \mathbf{u}_1)} \mathbf{u}_1 \\
&= (2, 0, 2) - \frac{0}{1}(0, 1, 0) \\
\mathbf{u}_2 &= (2, 0, 2) \\
\mathbf{u}_3 &= \mathbf{v}_3 - \frac{(\mathbf{v}_3 \cdot \mathbf{u}_1)}{(\mathbf{u}_1 \cdot \mathbf{u}_1)} \mathbf{u}_1 \\
&= (2, 3, 1) - \frac{3}{1}(0, 1, 0) \\
\mathbf{u}_3 &= \mathbf{v}_3 - \frac{(\mathbf{v}_3 \cdot \mathbf{u}_2)}{(\mathbf{u}_2 \cdot \mathbf{u}_2)} \mathbf{u}_2 \\
&= (2, 3, 1) - \frac{6}{8}(2, 0, 2) \\
\mathbf{u}_3 &= (0,5, 0, -0,5) \\
\text{Результат ортогонализации:}\\
\mathbf{u}_1 = (0, 1, 0) \\
\mathbf{u}_2 = (2, 0, 2) \\
\mathbf{u}_3 = (0,5, 0, -0,5) \\
\end{align*}
\section{Задача: Арифметические операции над матрицами}
\subsection{Исходные данные}
*:[(1, 2, 2) ,  (2, 3, 4)]; [(1, 2) ,  (1, 2),  (3, 4)]
\subsection{Решение}
\begin{align*}\\
mathbf{A} \times mathbf{B} = 
\begin{pmatrix}
1 \times 1 + 2 \times 1 + 2 \times 3\\
1 \times 2 + 2 \times 2 + 2 \times 4\\
2 \times 1 + 3 \times 1 + 4 \times 3\\
2 \times 2 + 3 \times 2 + 4 \times 4\\
\end{pmatrix}
text{=}
\begin{pmatrix}
9 & 14\\
17 & 26\end{pmatrix}
\\\end{align*}
\section{Задача: Арифметические операции над матрицами}
\subsection{Исходные данные}
*:[(1, 2, 2) ,  (2, 3, 4)]; 2
\subsection{Решение}
\begin{align*}\\
2 \times \mathbf{A} = 
\begin{pmatrix}
2 & 4 & 4\\
4 & 6 & 8\end{pmatrix}
\\\end{align*}
\section{Задача: Определитель матрицы}
\subsection{Исходные данные}
[(1, 2) ,  (2, 3)]
\subsection{Решение}
\begin{align*}
1 \times 3 +
2 \times 2 -
\det(\mathbf{A}) = -1
\end{align*}
\section{Задача: Обратная матрица}
\subsection{Исходные данные}
[(1, 2) ,  (2, 3)]
\subsection{Решение}
\begin{align*}
\det(\mathbf{A}) = -1\\
\text{Присоединённая матрица:}\\
\begin{pmatrix}
3 & -2\\
-2 & 1\end{pmatrix}
\\
\text{Обратная матрица:}\\
\begin{pmatrix}
-3 & 2\\
2 & -1\end{pmatrix}
\text{Проверка } \mathbf{A} \cdot \mathbf{A}^{-1} = \mathbf{I}:\\
\begin{pmatrix}
1 & 0\\
0 & 1\end{pmatrix}
\end{align*}
\section{Задача: Ранг матрицы}
\subsection{Исходные данные}
[(1, 2, 2) ,  (2, 3, 4)]
\subsection{Решение}
\begin{align*}
\text{Исходная матрица:}\\
\begin{pmatrix}
1 & 2 & 2\\
2 & 3 & 4\end{pmatrix}
\\
\text{Приводим к ступенчатому виду:}\\
\text{Меняем строки 1 и 2:}\\
\begin{pmatrix}
2 & 3 & 4\\
1 & 2 & 2\end{pmatrix}
\\
\text{Обнуляем строку 2 с помощью строки 1 (множитель 0,5):}\\
\begin{pmatrix}
2 & 3 & 4\\
0 & 0,5 & 0\end{pmatrix}
\\
\text{Ранг матрицы} = 2
\end{align*}
\section{Задача: Размер линейной оболочки}
\subsection{Исходные данные}
3:(0, 1, 0);(2, 0, 2);(2, 3, 1)
\subsection{Решение}
\begin{align*}
\text{Матрица из векторов (по столбцам):}\\
\begin{pmatrix}
0 & 2 & 2\\
1 & 0 & 3\\
0 & 2 & 1\end{pmatrix}
\\
\text{Размер линейной оболочки} = 3
\end{align*}
\section{Задача: Принадлежность линейной оболочке}
\subsection{Исходные данные}
2:(0, 1, 0);(2, 0, 2);(2, 3, 1)
\subsection{Решение}
\begin{align*}
\text{Проверяем принадлежность вектора } \mathbf{v} = \begin{pmatrix}0 \\ 1 \\ 0\end{pmatrix} \text{ линейной оболочке}\\
\text{Векторы оболочки:}\\
\begin{pmatrix}2 \\ 0 \\ 2\end{pmatrix}\\
\begin{pmatrix}2 \\ 3 \\ 1\end{pmatrix}\\
\text{Вектор не принадлежит линейной оболочке}
\end{align*}
\section{Задача: Решение СЛАУ}
\subsection{Исходные данные}
Жордан:[(2, 1) ,  (1, 3)]; [(4),  (5)]
\subsection{Решение}
\begin{align*}
\text{Расширенная матрица системы:}\\
\begin{pmatrix}
2 & 1 & 4\\
1 & 3 & 5\end{pmatrix}
\\
\text{Метод Жордана-Гаусса:}\\
\text{Нормируем строку 1 (делим на 2):}\\
\begin{pmatrix}
1 & 0,5 & 2\\
1 & 3 & 5\end{pmatrix}
\\
\text{Вычитаем строку 1 умноженную на 1 из строки 2:}\\
\begin{pmatrix}
1 & 0,5 & 2\\
0 & 2,5 & 3\end{pmatrix}
\\
\text{Нормируем строку 2 (делим на 2,5):}\\
\begin{pmatrix}
1 & 0,5 & 2\\
0 & 1 & 1,2\end{pmatrix}
\\
\text{Вычитаем строку 2 умноженную на 0,5 из строки 1:}\\
\begin{pmatrix}
1 & 0 & 1,4\\
0 & 1 & 1,2\end{pmatrix}
\\
\text{Решение системы:}\\
x_1 &= 1,4\\
x_2 &= 1,2\\
\end{align*}
\section{Задача: Собственные числа}
\subsection{Исходные данные}
[(2, 1) ,  (1, 3)]
\subsection{Решение}
\begin{align*}
\text{Характеристическое уравнение: } \det(\mathbf{A} - \lambda\mathbf{I}) = 0\\
\text{Используем QR-алгоритм для нахождения собственных чисел}\\
\text{Собственные числа:}\\
\lambda_1 &\approx 3,618\\
\lambda_2 &\approx 1,382\\
\end{align*}
\section{Задача: Собственные векторы}
\subsection{Исходные данные}
[(2, 1) ,  (1, 3)]
\subsection{Решение}
\begin{align*}
\text{Найденные собственные числа:}\\
\lambda_1 &= 3,618\\
\lambda_2 &= 1,382\\
\text{Соответствующие собственные векторы:}\\
\text{Для }\lambda_1 = 3,618:\\
\begin{pmatrix}
0,526 \\ 0,851\end{pmatrix}\\
\text{Для }\lambda_2 = 1,382:\\
\begin{pmatrix}
-0,851 \\ 0,526\end{pmatrix}\\
\end{align*}
\section{Задача: Уравнения прямой на плоскости}
\subsection{Исходные данные}
1; 2; 3
\subsection{Решение}
\begin{align*}
\text{Исходное уравнение прямой: }
1x + 2y + 3 = 0\\
\text{Уравнение с угловым коэффициентом: }
y = -0,5x +1,5\\
\text{Каноническое уравнение: }
\frac{x}{2} = \frac{y}{-1} = t\\
\end{align*}
\section{Задача: Точка пересечения прямых}
\subsection{Исходные данные}
1; 2; 3; 2;3;4
\subsection{Решение}
\begin{align*}
\text{Уравнения прямых: }
1: 1x + 2y + 3 = 0\\
2: 2x + 3y + 4 = 0\\
\text{Точка пересечения: }
\left(1, -2\right)\\
\end{align*}
\section{Задача: Расстояние от точки до прямой}
\subsection{Исходные данные}
(1, 1, 1);(2, 2, 2);(3, 3, 3)
\subsection{Решение}
\begin{align*}
\text{Прямая задана точкой } \mathbf{P}_0 = \begin{pmatrix}1 \\ 1 \\ 1\end{pmatrix} \text{ и направляющим вектором } \mathbf{v} = \begin{pmatrix}2 \\ 2 \\ 2\end{pmatrix}\\
\text{Точка } \mathbf{P} = \begin{pmatrix}3 \\ 3 \\ 3\end{pmatrix}\\
\text{Расстояние: }
d = \frac{|\mathbf{P_0P} \times \mathbf{v}|}{|\mathbf{v}|} = 0\\
\end{align*}
\section{Задача: Симметричная точка относительно прямой}
\subsection{Исходные данные}
(1, 1, 1);(2, 2, 2);(3, 3, 3)
\subsection{Решение}
\begin{align*}
\text{Исходная точка } \mathbf{P} = \begin{pmatrix}3 \\ 3 \\ 3\end{pmatrix}\\
\text{Проекция точки: } \mathbf{P}' = \begin{pmatrix}3 \\ 3 \\ 3\end{pmatrix}\\
\text{Симметричная точка: } \mathbf{P}'' = \begin{pmatrix}3 \\ 3 \\ 3\end{pmatrix}\\
\end{align*}
\section{Задача: Уравнения плоскости}
\subsection{Исходные данные}
1;2;3;4
\subsection{Решение}
\begin{align*}
\text{Общее уравнение плоскости: }
1x + 2y + 3z + 4 = 0\\
\text{Уравнение в отрезках: }
\frac{x}{-4} + \frac{y}{-2} + \frac{z}{-1,333} = 1\\
\text{Нормальное уравнение: }
\frac{1x + 2y + 3z + 4}{3,742} = 0\\
\end{align*}
\section{Задача: Уравнения прямой в n-мерном пространстве}
\subsection{Исходные данные}
(1, 1, 1);(2, 2, 2)
\subsection{Решение}
\begin{align*}
\text{Параметрические уравнения прямой: }\\
x_1 = 1 + 2t\\
x_2 = 1 + 2t\\
x_3 = 1 + 2t\\
\text{Канонические уравнения: }
\frac{x - 1}{2} = \frac{y - 1}{2} = \frac{z - 1}{2}\\
\end{align*}
\section{Задача: Пересечение плоскостей}
\subsection{Исходные данные}
1;2;3;4;1;2;3;4
\subsection{Решение}
\begin{align*}
\text{Уравнения плоскостей: }
1: 1x + 2y + 3z + 4 = 0\\
2: 1x + 2y + 3z + 4 = 0\\
\text{Прямая пересечения: }
\text{Параметрические уравнения: }
x = 0 + 0t\\
y = 0 + 0t\\
z = 0 + 0t\\
\text{Канонические уравнения: }
\frac{x - 0}{0} = \frac{y - 0}{0} = \frac{z - 0}{0}\\
\end{align*}
\section{Задача: Проекция прямой на плоскость}
\subsection{Исходные данные}
4;1;2;3;(1, 1, 1);(2, 2, 2)
\subsection{Решение}
\begin{align*}
\text{Уравнение плоскости: }
4x + 1y + 2z + 3 = 0\\
\text{Исходная прямая: }
\mathbf{r} = \begin{pmatrix}1 \\ 1 \\ 1\end{pmatrix} + t\begin{pmatrix}2 \\ 2 \\ 2\end{pmatrix}\\
\text{Проекция прямой: }
\mathbf{r} = \begin{pmatrix}-0,429 \\ -0,429 \\ -0,429\end{pmatrix} + t\begin{pmatrix}-0,667 \\ 1,333 \\ 0,667\end{pmatrix}\\
\end{align*}
\end{document}
